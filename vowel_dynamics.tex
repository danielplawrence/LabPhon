\documentclass[a0,final]{a0poster}
%%%Load packages
\usepackage{tipa}
\usepackage{amsmath}
\usepackage{enumerate}
\usepackage{graphicx}
\usepackage{float}
\usepackage[scriptsize]{subfigure}
\usepackage{caption}
\usepackage{multirow}
\usepackage{color}
\usepackage{natbib}
\usepackage{ragged2e}
\usepackage{multicol} 			%3-column layout
\usepackage[left=3cm,right=3cm,bottom=0cm,top=0cm]{geometry}			%Reset margins
\usepackage{helvet}				%Load Helvetica font & CM math
\usepackage{color}				%Needed for colour boxes & coloured text
\usepackage{graphics}
\newlength{\mylen}
\setbox1=\hbox{$\bullet$}\setbox2=\hbox{\small$\bullet$}
\setlength{\mylen}{\dimexpr0.75\ht1-0.5\ht2}
\renewcommand\labelitemi{\raisebox{\mylen}{\small$\bullet$}}

%%%Define colours and lengths
\definecolor{headingcol}{rgb}{1,1,1}	%Colour of main title
\definecolor{boxcol}{rgb}{0.7,0.2,0.2}		%Edge-colour of box and top banner
\fboxsep=1cm							%Padding between box and text
\setlength{\columnsep}{1cm}				%Set spacing between columns
\renewcommand{\familydefault}{\sfdefault}	%Set main text to sans-serifb
%%%Format title
\makeatletter							%Needed to include code in main file
\renewcommand\@maketitle{%
\null									%Sets position marker
{
\vspace*{8cm}
\color{headingcol}\sffamily\Huge		%Set title font and colour
\@title \par}%
\vskip 1em%
{
\color{white}\sffamily\LARGE				%Set author font and colour
\lineskip .5em%
\begin{tabular}[t]{l}%
\@author
\end{tabular}\par}%
\vskip 1cm
\par
}
\setlength{\parskip}{0cm}
\setlength{\parindent}{1em}
\makeatother

\title{\Huge{Vowel Dynamics and Social Meaning in York, Northern England}}

\author{Daniel Lawrence\\The University of Edinburgh\\\hspace{0.5cm}daniel.lawrence@ed.ac.uk}

\begin{document}
\hspace{-4cm}								%Align with edge of page, not margin
\vspace{-2cm}
\includegraphics{Black_Landscape.pdf}

%\colorbox{boxcol}{							%Coloured banner across top
\begin{minipage}{1191mm}					%Minipage for title contents
\vspace{-18cm}
\maketitle
\end{minipage}
%}
\vspace{.5cm}

\begin{multicols}{4}							%Use 3-column layout
						%Don't stretch contents vertically
\section*{Introduction}
\begin{itemize}
 \item{As time-varying acoustic events, speech sounds offer a wide range of variable cues which could potentially attach to the social meanings available in a speech community.} 

 \item{However, research into the social perception of phonetic variation has typically focused on `static' properties of speech events -- for example, by testing listeners' ability to use variation in average formant frequencies as a cue to social identity (Fridland, Bartlett \& Kreuz, 2004).} 

 \item{To address this gap, the present study explored listeners' social perceptions of the \textsc{goat} vowel (\textipa{/o/}) in York, Northern England, with a view to discovering \begin{itemize}\item{}how variable patterns of fronting and diphthongization might be available as indexical cues in this community.}
\end{itemize}
\section*{Data}
\begin{itemize}
\item{52 sociolinguistic interviews conducted in York, Northern England.}
\item{Social perception data from the same individuals.}
\end{itemize}
\vspace*{0.5cm}
\begin{table}[H]
\centering
\begin{tabular}{l|l|l}
Birth year&Female & Male \\
1935-1960 &7 &5\\
 1961-1980& 8 & 11\\
1981-2000& 10 &11\\
\end{tabular}
  \end{table}
\section*{Fronting and diphthongization in York}
\begin{itemize}
\item{Change in /o/ involves the interaction of fronting and diphthongization}
\item{Older speakers produced either a back diphthong or back monophthong}
\item{Younger speakers produce a front diphthong OR a back monophthong}
\item{A minority of younger speakers front /o/ primarily at the offglide, resulting in an upgliding diphthong}
\end{itemize}
\begin{itemize}
\item{GAM models:}
\item{F1-F2 trajectory plotted by decade}
\item{F2 intercepts plotted by YOB}
\item{F2 slope vs intercept}
\end{itemize}
\subsection*{Social factors}
\begin{itemize}
\item{Somehow show social factors (trajectory plots?)}
\end{itemize}
\columnbreak
\section*{Experimental design}
\section*{Social-indexical perception as signal detection}
\columnbreak
\section*{Results}
\section*{Conclusion}
\subsubsection*{References}
\vspace*{-.5cm}
\scriptsize
\begin{description}
\item[Campbell-Kibler, K. (2008).]{I'll be the judge of that: Diversity in social perceptions of (ING). \textit{Language in Society}, 37(05), 637-659.}\vspace*{0.2cm}

\item[Campbell-Kibler, K. (2009).]{The nature of sociolinguistic perception. Language Variation and Change, 21(01), 135-156.}

\item[Eckert, P. (2008).]{Variation and the indexical field. \textit{Journal of sociolinguistics}, 12(4), 453-476.}\vspace*{0.2cm}

\item[Fridland, V., Bartlett, K., \& Kreuz, R. (2004).]{Do you hear what I hear? Experimental measurement of the perceptual salience of acoustically manipulated vowel variants by Southern speakers in Memphis, TN. \textit{Language Variation and Change}, 16(01), 1-16.}\vspace*{0.2cm}

\item[Grosvald, M. (2009).]{Interspeaker variation in the extent and perception of long-distance vowel-to-vowel coarticulation. \textit{Journal of Phonetics}, 37(2), 173-188.}

\item[Haddican, B., Foulkes, P., Hughes, V., \& Richards, H. (2013).]{Interaction of social and linguistic constraints on two vowel changes in northern England. \textit{Language Variation and Change}, 25(03), 371-403.}

\item[Levon, E., \& Fox, S. (2014).]{ Social Salience and the Sociolinguistic Monitor A Case Study of ING and TH-fronting in Britain. \textit{Journal of English Linguistics}, 42(3), 185-217.}

\item[Munson, B. (2007).]{The acoustic correlates of perceived masculinity, perceived femininity, and perceived sexual orientation. \textit{Language and Speech}, 50(1), 125-142.}

\item[Purnell, T., Idsardi, W., \& Baugh, J. (1999).]{ Perceptual and phonetic experiments on American English dialect identification. \textit{Journal of Language and Social Psychology}, 18(1), 10-30.}


\end{description}
%\bibliographystyle{plain}
%\bibliography{halobib}

\end{multicols}
\end{document}
